\chapter{Comparing galaxy morphology and star formation properties in
  X-ray bright and faint groups and cluster}
\label{chap:xray}

\section{Introduction}
\label{sec:intro_x}

Numerous studies have shown a strong environmental dependence on the
star-forming and morphological properties of galaxies
\citep[e.g.][]{butcher1978, dressler1980, postman1984, dressler1999,
  blanton2005b, wetzel2012}.  Low-density regimes tend to be dominated
by star-forming, late-type galaxies whereas high-density areas, such
as galaxy clusters, tend to be primarily populated by quiescent,
early-type galaxies.  Within individual clusters, galaxy morphologies
tend to distribute as a function of local density (or equivalently
cluster-centric radius), with high fractions of late-type galaxies
being found at large radii and the regions near the cluster core being
dominated by early-types \citep[e.g.][]{dressler1980, postman1984,
  postman2005}.  This effect has become known as the
morphology-density relation.  While galaxies tends to distribute based
on their star-forming and morphological properties, the mechanism(s)
responsible for the quenching of star formation and morphological
transformations in galaxies are not well constrained -- although many
have been proposed.  Both mergers and impulsive galaxy-galaxy
interactions (`harassment') \citep[e.g.][]{moore1996} can induce
starburst events in galaxies leading to rapid consumption of gas
reserves and star formation quenching.  Within the virial radius of a
group or cluster the stripping of gas from galaxies becomes
efficient.  Both the stripping of hot halo gas (`strangulation')
\citep[e.g.][]{kawata2008} and cold gas stripping due to a dense
intracluster medium (`ram-pressure') \citep[e.g.][]{gunn1972} can
quench star formation.  As well, tidal interactions can affect gas
reservoirs by transporting gas from the galactic halo outwards which
subsequently allows it to more easily be stripped from the galaxy
\citep{chung2007}.
\par
On top of these environmental quenching mechanisms, previous authors
have found that secular processes, which depend on galaxy mass, appear
to play a significant role in star formation quenching
\citep{balogh2004, muzzin2012}.  The emergent picture for star
formation quenching appears to be some combination of environmental
quenching mechanisms and internal, secular processes.  In particular,
\citep{peng2010} suggests that in the low-redshift Universe,
environmental quenching is dominant for galaxies with $M_\star \lesssim
10^{10.5}\Msun$, whereas for galaxies with $M_\star \gtrsim
10^{10.5}\Msun$ mass quenching plays the more important role.
\par
While environmental and mass quenching within individual haloes are
seemingly strong effects, it is important to realize that groups and
clusters are not isolated structures.  In particular, galaxies can be
pre-quenched in group haloes prior to infall into a larger cluster.
This `pre-processing' suggests that many galaxies may already be
quenched upon cluster infall.  Simulations have shown that between
$\sim 25$ and $45$ per cent of infalling cluster galaxies may have
been pre-processed \citep{mcgee2009, delucia2012}.  Observationally,
\citep{hou2014} find that $\sim 25$ per cent fo the infall population
reside in subhaloes for massive clusters ($M_H \gtrsim 10^{14.5}\Msun$).
This pre-quenching of galaxies in groups could potentially be driven
by galaxy interactions and mergers which are favoured in the group
regime as a result of lower relative velocities between member
galaxies \citep{barnes1985, brough2006}.
\par
An important method for study the quenching mechanisms in groups and
clusters is to study the dependence of the star formation and
morphological properties of galaxies on the conditions of their host
halo (e.g.\ halo mass, X-ray luminosity, etc.).  In particular, if
quenching mechanisms depend on the density of the intra-group/cluster
medium (IGM/ICM) -- for example, ram-pressure stripping of cold gas --
then one would expect to see galaxy populations which are
preferentially passive in haloes with high X-ray luminosities.  Such
correlations have been looked for in previous studies, primarily
within cluster environments.
\par
In particular, \citet{ellingson2001} find no positive correlation
between the fraction of old galaxies and X-ray gas density.
\citet{balogh2002a} conclude that the level of star formation found in
their `low-$L_X$' sample is consistent with the levels seen in their
CNOC1 sample consisting of higher mass clusters.  \citet{fairley2002}
and \citet{wake2005} both study the fractions of blue galaxies at
intermediate redshifts and find no discernible trend between blue
fraction and X-ray luminosity.  Using multivariate regression
\citet{popesso2007b} find that cluster star formation depends on
cluster richness but find no additional dependence on X-ray
luminosity.  In addition, they find no significant correlation between
star-forming fraction and any global cluster property ($M_{200}$,
$\sigma_v$, $N_\mathrm{gal}$, and $L_X$).  \citet{lopes2014} find no
dependence of blue fraction on X-ray luminosity and the only slight
dependence they find between disc fraction and X-ray luminosity is
within the central and most dense regions.
\par
Conversely, \citet{balogh2002b} find that galaxies in their
`low-$L_X$' sample have preferentially high disc fractions compared to
galaxies in their `high-$L_X$' sample.  \citet{postman2005} find that
the bulge-dominated fraction for galaxies in high X-ray luminosity
clusters is higher than for those in low X-ray luminosity clusters.
In contrast with their star formation results, \citet{popesso2007b} do
find a significant anticorrelation between blue fraction and X-ray
luminosity.  Finally, \citet{urquhart2010} find an anticorrelation
between blue fraction and X-ray temperature for galaxies in
intermediate redshift clusters.
\par
In this paper we revisit the dependence of galaxy star formation and
morphological properties on the X-ray luminosity of the host halo.
Specifically, as a result of the large SDSS X-ray sample presented in
\citet{wang2014}, we are able to control for stellar mass, halo mass,
and radial dependences through fine-binning of the data set.  This
allows us to more directly investigate the effect of X-ray luminosity
on galaxies in different environments.
\par
The results of this study are presented as follows.  In
Section~\ref{sec:data_x} we briefly describe the SDSS group catalogues
utilized in this work, as well as the star formation and morphology
catalogues which we match to the group data set.  In
Section~\ref{sec:results_x} we present the primary results of this
paper, specifically, the differences between star-forming and
morphological trends in environments with different X-ray
luminosities.  In Section~\ref{sec:discussion_x} we provide a
discussion of the results presented in this paper.  Finally, in
Section~\ref{sec:conclusions_x} we provide a summary of the key
results and make concluding statements.
\par
In this paper we assume a $\Lambda$ cold dark matter cosmology with
$\Omega_M = 0.3$, $\Omega_\Lambda = 0.7$, and $H_0 = 70\,\mathrm{km}\,\mathrm{s^{-1}}\,\mathrm{Mpc^{-1}}$.

\section{Data}
\label{sec:data_x}

\subsection{Yang group catalogue}

This work relies heavily on the group catalogue of \citet{yang2007}.
The Yang group catalogue is constructed by applying the iterative
halo-based group finder of \citet{yang2005, yang2007} to the New York
University Value-Added Galaxy Catalogue (NYU-VAGC;
\citealt{blanton2005a}), which is based on the Sloan Digital Sky
Survey Data Release 7 (SDSS-DR7; \citealt{abazajian2009}).  The Yang
group catalogue has a wide range of halo masses, spanning from $\sim
10^{12}\Msun$ to $\sim 10^{15}\Msun$.  The catalogue contains both
objects which would be classified as groups ($10^{12} \lesssim M_H
\lesssim 10^{14}$) and as clusters ($M_H \gtrsim 10^{14}\Msun$),
however for brevity we will refer to all systems as groups regardless
of mass.
\par
Groups are initially populated using the traditional
friends-of-friends (FOF) algorithm \citep[e.g.][]{huchra1982}, as well
as assigning galaxies not yet linked to FOF groups as the centres of
potential groups.  Next, the characteristic luminosity, $L_{19.5}$,
defined as the combined luminosity of all group members with
$^{0.1}M_r - 5\log h \le -19.5$, is calculated for each group.  Using
the value of $L_{19.5}$ along with an assumption for the group
mass-to-light ratio, $M_H/L_{19.5}$, a tentative halo mass is assigned
on a group-by-group basis.  The tentative halo mass is used to
calculate a virial radius and velocity dispersion for each group,
which are then used to add or remove galaxies from the system.
Galaxies are assigned to groups under the assumption that the
distribution of galaxies in phase space follows that of dark matter
particles -- the distribution of which is assumed to follow a
spherical NFW profile \citep{navarro1997}.  This process is iterated
until the group memberships no longer change.
\par
Final halo masses given in the Yang group catalogue are determined
using the ranking of the characteristic stellar mass,
$M_{\star,\mathrm{grp}}$, and assuming a relationship between $M_H$
and $M_{\star,\mathrm{grp}}$ \citep{yang2007}.
$M_{\star,\mathrm{grp}}$ is defined by Yang et al. as

\begin{equation}
  M_{\star,\mathrm{grp}} = \frac{1}{g(L_{19.5},\,L_\mathrm{lim})}
  \sum_i \frac{M_{\star,i}}{C_i},
\end{equation}

\noindent
where $M_{\star,i}$ is the stellar mass of the $i$th member galaxy,
$C_i$ is the completeness of the survey at the position of that
galaxy, and $g(L_{19.5},\,L_\mathrm{lim})$ is a correction factor
which accounts for galaxies missed due to the magnitude limit of the
survey.  The statistical error in $M_H$ is on the order of
$0.3\,\mathrm{dex}$ and mostly independent of halo mass
\citep{yang2007}.

\subsection{SDSS X-ray catalogue}

To study the X-ray properties of the group sample, we utilize the SDSS
X-ray catalogue of \citet{wang2014}, which combines ROSAT All Sky
Survey (RASS) X-ray images in conjunction with optical groups
identified from SDSS-DR7 \citep{yang2007} to estimate X-ray
luminosities around $\sim 65\,000$ spectroscopic groups.
\par
To identify X-ray luminosities for individual groups, the algorithm of
\citet{shen2008} is employed.  Beginning from an optical group, the
most massive galaxies (MMGs) of that group are identifed -- up to four
MMGs are kept.  The RASS field in which the MMGs reside are then
identified, and an X-ray source catalogue is generated in the
$0.5-2.0\,\mathrm{keV}$ band \citep{wang2014}.  The maximum X-ray
emission density point is used to identify the X-ray centre of the
group, and any X-ray sources not associated with the group
(e.g.\ point source quasars or stellar object cross-matched from RASS
and SDSS-DR7), within one virial radius , are masked out.  Values for
the X-ray background, centred on the X-ray centre, are determined and
subtracted off and the X-ray luminosity, $L_X$, is calculated by
integrating the source count profile within the X-ray radius.
\par
Determining X-ray luminosities in this manner is susceptible to
`source confusion'.  Due to projection it is possible for more than
one group to contribute to the X-ray emission within the X-ray radius,
leading to an overestimation of the X-ray luminosity for a given
group.  To account for this effect \citet{wang2014} calculate the
`expected' average X-ray flux, $F_{X,i}$, for each group using the
average $L_X - M_H$ relation taken from \citet{mantz2010}.  They then
calculate the sum of the expected fluxes from each group for
multigroup systems and determine the contribution fraction,
$f_{\mathrm{mult},i}$, for each group defined as

\begin{equation}
  f_{\mathrm{mult},i} = F_{X,i} / \Sigma_i F_{X,i}.
\end{equation}

\noindent
The contribution factor will approximate the fraction of the observed
X-ray luminosity intrinsic to the individual group in question,
therefore applying this fraction to each group will act to debias the
measured X-ray luminosity from source confusion contamination.
\par
Within the Wang catalogue 817 groups have $\mathrm{S/N} > 3$, compared
to the total of $34\,522$ groups with positive detections (positive
count rates after background subtraction) and $\mathrm{S/N} > 0$.  We
run our analysis for groups with $\mathrm{S/N} > 3$ as well as groups
with $\mathrm{S/N} > 0$ and find that our choice of signal-to-noise
cut does not change the trends that we observe, therefore we focus on
the total sample ($\mathrm{S/N} > 0$) to ensure a sample size which is
large enough to finely bin the data in various properties
simultaneously.

\subsection{Final data set}

\begin{figure}[!ht]
  \centering
  \includegraphics[width=\textwidth]{mh_lx_con.pdf}
  \caption{Density contours for log X-ray luminosity versus log halo
    mass.  Dashed line corresponds to the linear least-squares
    best-fitting relationship.}
  \label{fig:mh_lx_con}
\end{figure}

\begin{figure}[!ht]
  \centering
  \includegraphics[width=0.8\textwidth]{data_smooth_lx2.pdf}
  \caption{Smoothed distributions for halo mass and X-ray luminosity
    within the sample.  Distributions are shown for both the X-ray
    strong (red, dashed) and the X-ray weak (blue, solid) samples.
    Shaded regions correspond to $2\sigma$ confidence intervals
    obtained from random bootstrap resampling.}
  \label{fig:data_smooth_lx2}
\end{figure} 

To obtain the final data set, we match the Wang SDSS X-ray catalogue
to the Yang SDSS group catalogue, giving us both optical and X-ray
group properties for the sample.  To obtain individual galaxy
properties we further match the data set to various public SDSS
catalogues as follows.
\par
We utilize stellar masses given in the NYU-VAGC, which are computed
following the methodology of \citet{blanton2007}.
\par
To obtain star formation rates (SFRs) and specific star formation
rates ($\mathrm{SSFR} = \mathrm{SFR} / M_\star$) we match the
catalogue of \citet{brinchmann2004} to the sample.  SFRs given by
Brinchmann et al. are determined using emission line fluxes whenever
possible; however, in the case of no clear emission lines or
contamination from active galactic nuclei (AGNs), SFRs are determined
using the strength of the $4000\,\mathrm{\mathring{A}}$ break
($D_n4000$) \citep{brinchmann2004}.
\par
We obtain galaxy morphologies from the catalogue of
\citet{simard2011}.  Simard et al. perform two-dimensional bulge +
disc decompositions for over one million galaxies from the Legacy area
of the SDSS-DR7, using three different fitting models: a pure
S{\'e}rsic model, a bulge + disc model with a de Vaucouleurs ($n_b =
4$) bulge, and a bulge + disc model with a free $n_b$.  To distinguish
between discy and elliptical galaxies we utilize the galaxy S{\'e}rsic
index, $n_g$, from the pure S{\'e}rsic decomposition.  We also use the
$V_\mathrm{max}$ weights given by Simard et al. to correct for the
incompleteness of our sample.
\par
We calculate group-centric distances for each galaxy using the
redshift of the group and the angular separation between the galaxy
and the luminosity-weighted centre of its host group.  We normalize
all of the galaxy radii by the virial radius of the host group,
$R_{180}$, which we calculate as in \citet{yang2007}

\begin{equation}
  R_{180} =
  1.26\,h^{-1}\,\mathrm{Mpc}\,\left(\frac{M_H}{10^{14}\,h^{-1}\,\Msun}
  \right )^{1/3} (1 + z_g)^{-1},
\end{equation}

\noindent
where $z_g$ is the redshift of the group centre.
\par
The final data set includes groups with halo masses ranging between
$10^{13} - 10^{15}\,\Msun$, and galaxies with stellar masses ranging
from $10^9 - 10^{11.3}\,\Msun$.  Group X-ray luminosities in the data
set are between $10^{39.6} -
10^{46.4}\,\mathrm{erg}\,\mathrm{s^{-1}}$, with a median value of
$10^{43.9}\,\mathrm{erg}\,\mathrm{s^{-1}}$, and are strongly
correlated with halo mass (see Fig.~\ref{fig:mh_lx_con}).  We do not
make an explicit radial cut, however over 99 per cent of member
galaxies fall within 1.5 virial radii.  Our final sample contains
$3\,902$ low-redshift ($z < 0.1$) groups hosting $41\,173$ galaxies.
The catalogue of \citet{wang2014} contains $\sim 35\,000$ groups.  The
fact that the final sample in this work is significantly smaller than
the original catalogue is twofold.  First, we restrict our sample to
redshifts smaller than 0.1 which reduces the number of groups from
$\sim 35\,000$ at $z < 0.2$ to $\sim 18\,000$ at $z < 0.1$.  The
second important cut is that we require $10^{13} < M_H < 10^{15}\Msun$
and a number of groups in the Wang catalogue have halo masses, $M_H <
10^{13}\Msun$ (where halo masses have been obtained from the catalogue
of \citealt{yang2007}).  This cut reduces the remaining number of
groups from $\sim 18\,000$ to $\sim 3\,900$.  It should be noted that
the majority of the $M_H < 10^{13}\Msun$ groups removed from the data
set are groups with very low membership.
\par
To determine the effect of X-ray luminosity on star formation and
morphology we consider two X-ray luminosity samples for the majority
of our analysis, which we refer to as the X-ray weak (XRW) and X-ray
strong (XRS) samples.  Similar to \citet{wang2014}, we define the XRS
sample to consist of all galaxies found below the $\log M_H - \log
L_X$ trend line.  This leads to an approximately equal number of
galaxies within the XRW and XRS samples.  We also performed our
analysis with a cut between the two X-ray samples at the median X-ray
luminosity of the data set, as well as defining the two samples using
the first and the fourth quartiles, however these alternative
definitions of the two X-ray samples do not change the trends that we
observe.
\par
Smoothed distributions for halo mass and X-ray luminosity are shown in
Fig.~\ref{fig:data_smooth_lx2} for both X-ray luminosity samples.
Density distributions are calculated using the \texttt{density}
\{\texttt{stats}\} function in the statistical computing language
\textsc{R} \citep{r2013}\footnote{http://www.R-project.org/} using a
Gaussian kernel.
\par
We study the dependence of star formation rates and morphology on
stellar mass by binning the data by stellar mass and calculating the
disc and star-forming fractions for each bin.  Binning by stellar mass
is important to account for the systematic dependence of star
formation and morphology on stellar mass
\citep[e.g.][]{brinchmann2004, whitaker2012}.  Additionally, as the
relative balance between environmental and mass quenching is not well
understood, it is important to investigate the effects of environment
at a given stellar mass.
\par
We define the star-forming fraction, $f_{SF}$, as the fraction of
galaxies in each bin with $\log \mathrm{SSFR} > -11$.
\citet{wetzel2012} show that at low redshift the division between the
red sequence and the blue cloud is found at $\log \mathrm{SSFR} \simeq
-11$ across a wide range of halo masses.  For each stellar mass bin
the star-forming fraction is given by

\begin{equation}
  f_{SF} =
  \frac{V_\mathrm{max}\;\mathrm{weighted}\;\mathrm{no.}\;\mathrm{of}\;\mathrm{galaxies}\;\mathrm{with}\;\log
    \mathrm{SSFR} >
    -11}{V_\mathrm{max}\;\mathrm{weighted}\;\mathrm{total}\;\mathrm{no.}\;\mathrm{of}\;\mathrm{galaxies}}
\end{equation}

\noindent
Similarly we define the disc fraction, $f_D$, as the fraction of
galaxies in each bin with S{\'e}rsic index, $n < 1.5$.  For each
stellar mass bin this is given by

\begin{equation}
  f_{SF} =
  \frac{V_\mathrm{max}\;\mathrm{weighted}\;\mathrm{no.}\;\mathrm{of}\;\mathrm{galaxies}\;\mathrm{with}\;
    n < 1.5}{V_\mathrm{max}\;\mathrm{weighted}\;\mathrm{total}\;\mathrm{no.}\;\mathrm{of}\;\mathrm{galaxies}}
\end{equation}

\noindent
We also ran our analysis using a dividing cut at S{\'e}rsic indices of
$n=1.0$ and $n=2.0$ to define a disc galaxy, however using these
alternative definitions for a disc galaxy does not alter the trends
that we observe.

\section{Results}
\label{sec:results_x}

\subsection{Star-forming and morphology trends in strong and weak
  $L_X$ samples}

\begin{figure}[!ht]
  \centering
  \includegraphics[width=\textwidth]{disk_sfFrac_w_lx1234.pdf}
  \caption{Left: star-forming fraction versus stellar mass for the
    four X-ray luminosity quartiles of the data sample.  Right: disc
    fraction versus stellar mass for the four X-ray luminosity
    quartiles of the sample.  Error bars correspond to $1\sigma$
    Bayesian binomial confidence intervals given in \citet{cameron2011}}
  \label{fig:disk_sfFrac_w_lx1234}
\end{figure}

To investigate the effect of X-ray luminosity on galaxy properties, in
Fig.~\ref{fig:disk_sfFrac_w_lx1234} we show star-forming and disc
fractions, as a function of stellar mass, for subsamples corresponding
to the four X-ray luminosity quartiles of the data set.  Examination
of Figs~\ref{fig:disk_sfFrac_w_lx1234}(a) and (b) show that
star-forming and disc fractions follow a consistent marching order
with respect to X-ray luminosity.  The disc and star-forming fractions
decrease as X-ray luminosity increases.

\begin{figure}[!ht]
  \centering
  \includegraphics[width=\textwidth]{disk_sfFrac_w_lx2_mh.pdf}
  \caption{Star-forming (solid lines) and disc (dashed lines)
    fractions versus stellar mass, for different halo mass bins and
    the XRW (blue) and XRS (red) samples.  Error bars correspond to
    $1\sigma$ Bayesian binomial confidence intervals given in \citet{cameron2011}}
  \label{fig:disk_sfFrac_w_lx2_mh}
\end{figure}

We note that the results in Fig.~\ref{fig:disk_sfFrac_w_lx1234}
consider all halo masses in the sample, however it has been found that
galaxy morphology and star formation depend on local density and halo
mass \citep{dressler1980, balogh2004, wetzel2012, lackner2013}
(however also see: \citealt{delucia2012, hoyle2012, hou2013}).  As
shown in Fig.~\ref{fig:mh_lx_con} the data show a strong correlation
between X-ray luminosity and halo mass, therefore we must determine if
differences shown in Fig.~\ref{fig:disk_sfFrac_w_lx1234} are simply a
result of galaxies in higher $L_X$ environments being housed in
preferentially high-mass haloes.
\par
To control for any potential halo mass effect, we further bin the data
into narrow halo mass bins and re-examine the dependence of galaxy
properties on X-ray luminosity, considering now the XRW and XRS
samples form Fig.~\ref{fig:mh_lx_con}.
Fig.~\ref{fig:disk_sfFrac_w_lx2_mh} shows star-forming (solid) and
disc (dashed) fractions as a function of stellar mass for four
different halo mass bins -- ranging from $10^{13}$ to $10^{15}\Msun$
with bin widths of $0.5\,\mathrm{dex}$.  Data are binned according to
stellar mass and markers are plotted at the median bin values.  For
each halo mass bin we show star-forming and disc fractions from the
X-ray strong and X-ray weak samples.
\par
For both star-forming and disc fractions we continue to see a residual
trend with X-ray luminosity, even after controlling for any halo mass
dependence: star-forming and disc fractions are systematically higher
in the XRW sample.  We see the strongest trends in the intermediate
and high-mass haloes.  The difference between the strong (red) and
weak (blue) X-ray luminosity samples is clearest at low stellar mass,
and in all haloes the two samples converge at moderate to high stellar
mass.

\subsection{Radial dependence of star-forming and morphology trends}

\begin{figure}[!ht]
  \centering
  \includegraphics[width=\textwidth]{disk_sfFrac_w_lx2_mh_rxh.pdf}
  \caption{Star-forming (solid lines) and disc (dashed lines)
    fractions versus stellar mass, for galaxies outside of their host
    X-ray radius and for different halo mass bins and
    the two $L_X$ samples.  Error bars correspond to
    $1\sigma$ Bayesian binomial confidence intervals given in \citet{cameron2011}}
  \label{fig:disk_sfFrac_w_lx2_mh_rxh}
\end{figure}

\begin{figure}[!ht]
  \centering
  \includegraphics[width=\textwidth]{disk_sfFrac_w_lx2_mh_rxl.pdf}
  \caption{Same as Fig.~\ref{fig:disk_sfFrac_w_lx2_mh_rxh} for
    galaxies inside of their host X-ray radius.}
  \label{fig:disk_sfFrac_w_lx2_mh_rxl}
\end{figure}

Within host groups X-ray emission is concentrated at relatively small
group-centric radii, with X-ray emission generally extending out to
half a virial radius \citep{wang2014}.  If the trends we are observing
are a result of increased gas density, we would expect to see enhanced
trends (i.e.\ a larger difference between the XRS and XRW samples) at
small group-centric radii and suppressed trends at large radii.  To
test this we further divide the data into subsets corresponding to
those galaxies that lie within the X-ray emission radius (using the
X-ray radius, $R_\mathrm{Xray}$, given in \citealt{wang2014}) and
those galaxies that lie outside of the X-ray radius.  We again plot
star-forming/disc fraction versus stellar mass, in narrow halo mass
bins, for the large and small radius subsamples.  The results of this
analysis are shown in Figs~\ref{fig:disk_sfFrac_w_lx2_mh_rxh} and
\ref{fig:disk_sfFrac_w_lx2_mh_rxl}, where the two figures correspond
to disc fraction and star-forming fraction trends for the large and
small radius subsamples, respectively.
\par
Examination of Figs~\ref{fig:disk_sfFrac_w_lx2_mh_rxh} and
\ref{fig:disk_sfFrac_w_lx2_mh_rxl} shows that for both galaxies found
within their host halo's X-ray radius and those found outside, we
still see an increase in star-forming and disc fractions in the XRW
sample -- as before this effect is strongest in the intermediate-to
high-mass haloes and at low stellar mass.  Also the disc and
star-forming fractions tend to be higher at large radii, which is
consistent with the morphology-density relation.

\begin{figure}[!ht]
  \centering
  \includegraphics[width=\textwidth]{lxDiff_r.pdf}
  \caption{SF and disc excess versus stellar mass for both galaxies
    within (purple, solid) and outside (green, dashed) of the X-ray
    radius.  Panels a-d show SF excess for four halo mass bins and
    panels e-h show disc excess for four halo mass bins.  Shaded
    regions represent $1\sigma$ confidence intervals.}
  \label{fig:lxDiff_r}
\end{figure}

To further investigate if the increase in star-forming and disc
fractions in the XRW sample compared to the XRS sample -- which we
will refer to as the `SF excess' and the `disc excess' -- depends on
whether you consider galaxies within or outside of the X-ray radius,
we show SF and disc excess versus stellar mass in
Fig.~\ref{fig:lxDiff_r}.  We quantitatively define SF and disc excess
as

\begin{align}
  & \mathrm{SF}\;\mathrm{excess} = f_{SF}(\mathrm{XRW}) -
  f_{SF}(\mathrm{XRS}) \\
  & \mathrm{Disc}\;\mathrm{excess} = f_{D}(\mathrm{XRW}) -
  f_{D}(\mathrm{XRS})
\end{align}

\noindent
where $f_{SF}(\mathrm{XRW})$ and $f_{SF}(\mathrm{XRS})$ are the
star-forming fractions in the XRW and XRS samples respectively, and
analogously for $f_D(\mathrm{XRW})$ and $f_D(\mathrm{XRS})$.
\par
We find no radial dependence for SF and disc excess as the two radial
subsamples in Fig.~\ref{fig:lxDiff_r} show overlap for all halo and
stellar masses.  With the exception in Fig.~\ref{fig:lxDiff_r}(c) where
the SF excess, for low-mass galaxies, is stronger for galaxies within
the X-ray radius. 

\section{Discussion}
\label{sec:discussion_x}

\section{Summary \& Conclusions}
\label{sec:conclusions_x}

% Bibliography
%
\bibliographystyle{apj}
\bibliography{masters-thesis}
