The cycle of star formation is the key to galaxy evolution.  Stars form in massive collections of extremely dense cold gas.  Stellar feedback will inject turbulence into the interstellar medium (ISM) and regulate the availability of more star-forming gas.  This gas is an integral component in the cycle of star formation but is very difficult to model in numerical simulations.  We have investigated the interplay between star formation and the structure of the ISM in numerical simulations.  These simulations were done using the Smoothed Particle Hydrodynamics code \textsc{Gasoline}.  For this work we introduce a new treatment for photoelectric heating in \textsc{Gasoline}.  We first explore the impact of numerical parameter choices for the star formation threshold density ($n_{\text{th}}$), star formation efficiency ($c_*$) and feedback efficiency ($\epsilon_{\text{FB}}$).  Of these three parameters, only the feedback efficiency plays a large role in determining the global star formation rate of the galaxy.  Further, we explore the truncation of star formation in the outer regions of galactic discs and its relation to the presence of a two-phase thermal instability.  In the outer regions of the simulated discs, gas exists almost exclusively in one warm phase, unsuitable to host large-scale star formation.  We find that the disappearance of two-phase structure in the ISM corresponds to the truncation of star formation.


%In our simulations, the disappearance of star formation does correspond to regions of the disc where only a warm medium exists.  Here the surface density is too low
%
%%Stars form in massive collections of extremely dense cold gas.  This gas is an integral component in the cycle of star formation and the evolution of galaxies. Stars will 
%
%%These processes inject turbulence into the ISM and regulate the availability of more star-forming gas.
%providing the fuel to create stars which will 
%
%
%Stars form with low efficiency from massive collections of dense gas and dust.   
%
%This dense and cold star-forming gas is difficult to resolve in numerical simulations.  
%
%It is however an integral component in the cycle of star formation and is shaped by the stars it forms.  
%
%The interplay between star formation and the structure of the Interstellar Medium (ISM) is investigated using numerical simulations.  
%
%We have used a suite of simulated disc galaxies using the SPH code \textsc{Gasoline}.  
%
%We introduce a new treatment for photoelectric heating into \textsc{Gasoline}.   
%
%We study first the impact of numerical parameter choices for the star formation threshold density ($n_{\text{th}}$), star formation efficiency ($c_*$) and feedback efficiency ($\epsilon_{\text{FB}}$).  
%
%We find that only the feedback efficiency plays a large role in determining the total number of stars formed by a galaxy.  
%
%
%
%We find that the disappearance of star formation does correspond to regions of the disc where gas has low surface density and thus can sustain only a warm medium, unsuitable to host large-scale star formation.
%
%
%Star formation is a self-regulated cycle.

%Stars create turbulence.

%There is an intimate link between stars, the interstellar medium and galaxies as a whole.

%The interplay between star formation and the structure of the Interstellar Medium (ISM) is investigated using numerical simulations.  We explore the impact of star formation criteria on the global star formation rate and the structure of the ISM.  We explore the truncation of star formation in the outer regions of galactic disks and its relation to the presence of a two-phase thermal instability.

%We find that 
