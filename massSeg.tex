\chapter{Mass segregation trends in SDSS galaxy groups}
\label{chap:massSeg}

\section{Introduction}
\label{sec:intro_ms}

It has been well established that galaxy properties dpend strongly on
local environment \citep[e.g.][]{oemler1974, hogg2004, blanton2005b,
  tal2014}.  Galaxies in dense environments such as clusters tend to
have lower star formation rates (SFRs), while isolated field galaxies
are generally actively forming stars \citep[e.g.][]{balogh2000,
  ball2008, wetzel2012}.  It is also well known that galaxy
properties, such as SFR, depend strongly on galaxy mass
\citep[e.g.][]{poggianti2008}.  It is critical to study the
distribution of galaxy masses within haloes of different masses in
order to ascertain whether the variations in galaxy properties with
environment are due to physical mechanisms acting in dense
environments, or simply due to the fact that high-density environments
contain more high-mass galaxies.  Intermediate-density environments,
galaxy groups, represent not only the most common environment in the
local Universe \citep{geller1983, eke2005}, but also represent the
environment where many physical processes are efficient.  Galaxy
interactions such as mergers and harassment are favoured in this
environment because of the low relative velocities between galaxies
\citep{zabludoff1998, brough2006}.
\par
The study of mass segregation in groups can be used to elucidate
information on physical processes such as dynamical friction, galaxy
mergers, and tidal stripping.  Mass segregation in bound structures
has generally been predicted as a result of dynamical friction
\citep{chandrasekhar1943}.  Dynamical friction acts as a drag force on
orbiting bodies and massive galaxies within groups and clusters are
expected to migrate to smaller radii as time progresses.  If dynamical
friction is a dominant factor, then clear mass segregation should be
observed in evolved groups and clusters.
\par
Galaxy groups are not static systems, but are constantly being
replenished by infalling galaxies from the field.  Infalling galaxies
are preferentially found at large radii \citep{wetzel2013} and the
difference in stellar mass distributions between evolved group members
and infalling galaxies could affect the strength of mass segregation.
\par
If significant mass segregation is not found, then this implies that
either: the time-scale associated with dynamical friction is greater
than the age of the group/cluster, or that there are other physical
processes, such as merging, tidal stripping, or pre-processing, which
are playing a more important role than dynamical friction.
\par
Recent work has shown conflicting results with regards to the presence
of mass segregation in groups and clusters.  \citet{ziparo2013} find
no evidence for strong mass segregation in X-ray selected groups out
to $z=1.6$, for a sample of galaxies with $M_\mathrm{star} >
10^{10.3}\Msun$.  \citet{vonderlinden2010} examine Sloan Digital Sky
Survey (SDSS) galaxy clusters and find no evidence for mass
segregation in four different redshift bins at $z < 0.1$.  von der
Linden et al. make redshift-dependent stellar mass cuts ranging from
$10^{9.6}$ to $10^{10.5}\Msun$.  \citet{vulcani2013} use mass-limited
samples at $0.3 \le z \le 0.8$ from the IMACS Cluster Building Survey
and the ESO Distant Cluster Survey, with stellar mass cuts at
$M_\mathrm{star} > 10^{10.5}\Msun$ and $M_\mathrm{star} >
10^{10.2}\Msun$, respectively, to study galaxy stellar mass functions
in different environments.  Vulcani et al. find no statistical
differences between mass functions of galaxies located at different
cluster-centric distances.
\par
Conversely, \citet{balogh2014} find evidence for mass segregation in
Group Environment Evolution Collaboration 2 (GEEC2) groups at $0.8 < z
< 1$, using as stellar-mass-limited sample with $M_\mathrm{star} >
10^{10.3}\Msun$.  Using a volume limited sample of zCOSMOS groups,
\citet{presotto2012} find evidence for mass segregation in their whole
sample at both $0.2 \le z \le 0.45$ and $0.45 \le z \le 0.8$.
Presotto et al. also break their sample into rich and poor groups at
$0.2 \le z \le 0.45$, and find evidence for mass segregation within
rich groups but find no evidence for mass segregation within poor
groups.  Using a $V_\mathrm{max}$-weighted sample with a stellar mass
cut at $10^{9.0}\Msun$, \citet{vandenbosch2008a} find evidence for mass
segregation in SDSS groups.
\par
It is clear that there is no consensus regarding the strength of mass
segregation in groups and clusters or its halo mass dependence.
\par
In this Letter, we present evidence of the presense of a small, but
significant, amount of mass segregation in SDSS galaxy groups.  We
show that the detection of mass segregation is dependent on stellar
mass completeness, with completeness cuts at relatively high stellar
masses potentially masking underlying mass segregation trends.  We
also show that the strength of mass segregation scales inversely with
halo mass, with cluster-sized haloes showing little to no observable
mass segregation.  In Section~\ref{sec:data_ms}, we briefly describe
our data set, 
in Section~\ref{sec:results_ms} we present our results from this work,
in Section
... we provide a discussion of our results, and in Section ... we give
a summary of the results and make concluding statements.
\par
In this Letter, we assume a flat $\Lambda$ cold dark matter cosmology
with $\Omega_M = 0.3$, $\Omega_\Lambda = 0.7$, and $H_0 =
70\,\mathrm{km}\,\mathrm{s^{-1}}\,\mathrm{Mpc^{-1}}$.

\section{Data}
\label{sec:data_ms}

The results presented in this Letter utilize the group catalogue of
\cite{yang2007}.  This catalogue is contructed by applying the
halo-based group finder of \citet{yang2005, yang2007} to the New York
University Value-Added Galaxy Catalogue (NYU-VAGC;
\citealt{blanton2005a}), which is based on the SDSS Data Release 7
(DR7; \citealt{abazajian2009}.  Stellar masses are obtained from the
NYU-VAGC and are computed using the methodology of
\citet{blanton2007}, assuming a \citet{chabrier2003} initial mass
function.  Halo masses are determined using the ranking of the
characteristic stellar mass, $M_{\star,\mathrm{grp}}$, and assuming a
relationship between $M_\mathrm{halo}$ and $M_{\star,\mathrm{grp}}$
\citep{yang2007}.  $M_{\star,\mathrm{grp}}$ is defined by Yang et
al. as

\begin{equation}
  M_{\star,\mathrm{grp}} = \frac{1}{g(L_{19.5},\,L_\mathrm{lim})}
  \sum_i \frac{M_{\mathrm{star},i}}{C_i},
\end{equation}

\noindent
where $M_{\mathrm{star},i}$ is the stellar mass of the $i$th member
galaxy, $C_i$ is the completeness of the survey at the position of
that galaxy, and $g(L_{19.5},\,L_\mathrm{lim})$ is a correction factor
which accounts for galaxies missed due to the magnitude limit of the
survey.
\par
Halo-centric distance for each galaxy is not given explicitly in the
Yang catalogue; however, we calculate it using the redshift of the
group and the angular separation of the galaxy and halo centre on the
sky.  We measure group-centric radius from the luminosity-weighted
centre of each group, and normalize our grou-centric radii by
$R_{200}$.  We use the definition for $R_{200}$ as given in
\citet{carlberg1997}

\begin{equation}
  R_{200} = \frac{\sqrt{3} \sigma}{10 H(z)},
\end{equation}

\noindent
where the Hubble parameter, $H(z)$, is defined as

\begin{equation}
  H(z) = H_0\sqrt{\Omega_M (1+z)^3 + \Omega_\Lambda},
\end{equation}

\noindent
and we calculate the velocity dispersion, $\sigma$, as

\begin{equation}
  \sigma = 397.9\,\mathrm{km}\,\mathrm{s^{-1}} \left
(\frac{M_\mathrm{halo}}{10^{14}\,h^{-1}\Msun} \right )^{0.3214},
\end{equation}

\noindent
where the above is a fitting function given in \citet{yang2007}.
\par
For our analysis we select group galaxies with redshift, $z < 0.1$,
that are within two virial radii of the group centre, and groups with
a minimum of three galaxy members -- although our results are not
sensitive to these specific cuts.  For our sample over 95 per cent of
group galaxies reside within two virial radii of the group centre.  We
also subtract the most massive galaxy (MMG) from each group, to ensure
that any underlying radial mass trend is not contaminated by the MMG.
\par
This sample is not volume limited, therefore, the sample will suffer
from the Malmquist bias.  This leads to a bias towards objects of
higher luminosity and stellar mass, with increasing redshift.  To
account for this bias we weight our sample by $1/V_\mathrm{max}$,
where $V_\mathrm{max}$ is the comoving volume of the Universe out to a
comoving radius at which the galaxy would have met the selection
criteria for the sample.  For our $V_\mathrm{max}$ weights we apply
the values presented in the catalogue of \citet{simard2011} to our
sample.
\par
In order to investigate the effect of stellar mass limits on the
detection of mass segregation, we use samples corresponding to various
stellar mass cuts.  We perform our analysis on an unweighted sample
with two mass cuts corresponding to $M_\mathrm{star} > 10^{10.5}\Msun$
(4152 galaxies in 1970 groups) and $M_\mathrm{star} > 10^{10.0}\Msun$
(26 774 galaxies in 4534 groups); and a $V_\mathrm{max}$-weighted
sample with mass cuts at $M_\mathrm{star} > 10^{9.0}\Msun$
(56 957 galaxies in 7217 groups) and $M_\mathrm{star} > 10^{8.5}\Msun$
(59 791 galaxies in 7289 groups).  The unweighted sample is stellar
mass complete down to $M_\mathrm{star} > 10^{10.0}\Msun$.  Therefore,
for both the weighted and unweighted sample, we have two different
stellar mass cuts, giving us four separate samples in total.

\section{Results}
\label{sec:results_ms}

\subsection{Mass segregation in SDSS groups}

\begin{figure}[!ht]
  \centering
  \includegraphics[width=\textwidth]{Rmass.pdf}
  \caption[Mean mass versus group-centric radius for various halo mass
    bins]{All panels show mean mass as a function of normalized
    distance for various halo mass bins, with error bars corresponding
  to $1\sigma$ statistical errors.  The solid lines correspond to
  weighted least-squares fits for each halo mass bin.  Top left:
  unweighted sample, for galaxies with $\log(M_\mathrm{star}/\Msun) >
  10.5$.  Top right: unweighted sample, for galaxies with
  $\log(M_\mathrm{star}/\Msun) > 10.0$.  Bottom left:
  $V_\mathrm{max}$-weighted sample, for galaxies with
  $\log(M_\mathrm{star}/\Msun) > 9.0$.  Bottom right:
  $V_\mathrm{max}$-weighted sample, for galaxies with
  $\log(M_\mathrm{star}/\Msun) > 8.5$.  Note that different mass
  scales are used in each panel.  There are more halo mass bins in the
  bottom row due to the increased number of low-mass galaxies as a
  result of $V_\mathrm{max}$ weighting.}
  \label{fig:Rmass}
\end{figure}

In Fig.~\ref{fig:Rmass} we plot mean stellar mass as a function of
radial distance from the group centre for various halo mass bins.
Fig~\ref{fig:Rmass}(a) corresponds to our high-mass cut, unweighted
sample; Fig~\ref{fig:Rmass}(b) corresponds to our low-mass cut,
unweighted sample; Fig~\ref{fig:Rmass}(c) corresponds to our high-mass
cut, weighted sample; and Fig~\ref{fig:Rmass}(d) corresponds to our
low-mass cut, weighted sample.
\par
For all halo mass bins, and regardless of the mass cut, the unweighted
sample shows statistically significant mass segregation with a
weighted linear least-squares fit.  The $V_\mathrm{max}$-weighted
sample shows statistically significant mass segregation for the five
lower halo mass bins, whereas the highest halo mass bin has a
best-fitting slope consistent with zero -- this trend hold for both
mass cuts.  For both the weighted and unweighted samples there is a
clear trend of the slope with halo mass -- more massive haloes show
weaker mass segregation.  This result will be discussed in Section ...
\par
We find that our highest halo mass sample ($M_\mathrm{halo} >
10^{14.5}\Msun$) has a large number of low-mass galaxies when compared
to the high-halo-mass samples, which leads to a smaller mean stellar
mass in the $V_\mathrm{max}$-weighted results shown in
Figs~\ref{fig:Rmass}(c) and (d).  While this introduces a shift in
normalization, it does not affect the mass segregation trend and
therefore does not change the key result that mass segregation depends
on halo mass.

\subsection{Massive galaxy fraction}

\begin{figure}[!ht]
  \centering
  \includegraphics[width=\textwidth]{Rfrac.pdf}
  \caption[Fraction of massive galaxies versus group-centric radius
    for various halo mass bins]{Fraction of massive galaxies with respect to normalized
    radial distance.  Error bars are given by a $1\sigma$ binomial
    confidence interval, calculated using the beta distribution as
    outlined in \citet{cameron2011}.  The solid lines correspond to
    weighted least-squares fits for each halo mass bin.  Left-hand
    panel: the fraction of galaxies with $\log(M_\mathrm{star}/\Msun)
    > 10.25$ as a function of radial distance, for the unweighted
    sample with $M_\mathrm{star} > 10^{10}\Msun$.  Right-hand panel:
    the fraction of galaxies with $\log(M_\mathrm{star}/\Msun) > 10.5$
  as a function of radial distance, for the unweighted sample with
  $M_\mathrm{star} > 10^{10}\Msun$.}
  \label{fig:Rfrac}
\end{figure}

An alternative way to investigate galaxy populations within the group
sample is to study the fraction of `massive' galaxies at various
group-centric radii.  In Fig..., we plot the fraction of massive
galaxies as a function of radial distance for two different
definitions of what constitutes a massive galaxy.  We calculate the
massive fraction for each radial bin as

\begin{equation}
  f_m(M_\mathrm{cut}) =
  \frac{\#\;\mathrm{galaxies}\;\mathrm{with}\;M_\mathrm{star} >
    M_\mathrm{cut}}{\#\;\mathrm{galaxies}\;\mathrm{with}\;M_\mathrm{star}
    > 10^{10}\Msun},
\end{equation}

\noindent
where $M_\mathrm{cut}$ is a stellar mass cut-off above which we define
a massive galaxy.  We initially apply a high-mass galaxy cut,
$M_\mathrm{cut}$, at $10^{10.25}\Msun$, corresponding to the median
stellar mass of the unweighted sample (with the low-mass cut at
$10^{10}\Msun$).  Comparing Figs~\ref{fig:Rmass}(b) and
\ref{fig:Rfrac}(a) we see essentially identical trends.  We observe
the same trends of mass segregation whether we look at the average
galaxy mass at a given radius, or consider the fraction of massive
galaxies.
\par
To confirm that this trend is robust regardless of the mass cut-off
used to define a massive galaxy, we make the same plot but now use
$M_\mathrm{cut} = 10^{10.5}\Msun$.  Comparing Figs~\ref{fig:Rfrac}(a)
and (b) we see that while the overall fractions of massive galaxies
decrease with increasing the stellar mass cut, the trend essentially
stays the same.  There is clear evidence for mass segregation and the
strength of mass segregation depends on halo mass.

\section{Discussion}
\label{sec:discussion_ms}

\subsection{Effect of including low-mass galaxies}

The results in Fig.~\ref{fig:Rmass} show that mass segregation
generally increases when lower mass galaxies are included.  To
quantify this effect we can compare the best-fitting slopes
corresponding to the high-mass and the low-mass cut samples.
\par
For a given halo mass, the low-mass cut sample displays larger slopes
than the high-mass cut sample for two of the halo mass bins.  The
slopes corresponding to the other two halo mass bins are consistent
with being equal.  For the weighted samples we find similar results
with the low-mass cut sample showing larger slopes for three of the
halo mass bins, and the other three halo mass bins showing slopes
consistent with being equal.
\par
This suggests that the inclusion of low-mass galaxies has a measurable
effect on the observation of mass-segregation.  Studies which make
mass cuts at moderate to high-stellar mass, are potentially missing a
mass segregation contribution from low-mass galaxies.  The observation
of mild mass segregation is consistent with the low redshift sample of
\citet{ziparo2013}; however, they see similar mass-radius relations
regardless of the stellar mass cut applied.

\subsection{Halo mass dependence}

Figs~\ref{fig:Rmass} and \ref{fig:Rfrac} clearly indicate that the
highest halo mass bins show the least mass segregation.  This trend is
consistent in all cases, regardless of stellar mass cut or whether the
sample had $V_\mathrm{max}$ weights applied.  Our observed dependence
on halo mass is consistent with results finding no measurable mass
segregation in galaxy clusters \citep{pracy2005, vonderlinden2010,
  vulcani2013}
\par
It has been shown through $N$-body simulations that the dynamical
friction time-scale scales with $M_h/M_s$ \citep[e.g.][]{taffoni2003,
  conroy2007, boylan2008}, where $M_s$ is the initial satellite mass
and $M_h$ is the mass of the host halo.  For a given satellite mass,
this implies a longer dynamical friction time-scale for larger haloes,
which is consistent with our result.  This can be interpreted as an
increase in tidal stripping efficiency as $M_h / M_s$ increases
\citep{taffoni2003}.  \citet{gan2010} have shown that for an infalling
satellite the dynamical friction time-scale increases with a stronger
tidal field.  This is due to tidal stripping retarding the decay of
satellite angular momentum, which increases the dynamical friction
time-scale.
\par
It should be noted that the merger time-scale scales with $M_s / M_h$
\citet{jiang2008}, which implies a higher merger efficiency in
low-mass haloes, for a given satellite mass.  The build-up of massive
objects through galaxy mergers could enhance mass segregation in
low-mass haloes, in accordance with our results.
\par
There has been evidence of cluster galaxies having their star
formation quenched in lower mass groups ($\sim 10^{13}\Msun$) prior to
accretion into the cluster environment \citep[e.g.][]{zabludoff1998,
  mcgee2009, delucia2012, hou2014}.  This pre-processing could
potentially provide an explanation of our observed mass segregation
trends with halo mass.  If mass segregation is present in the group
environment as a result of pre-processing, the recent accretion of
multiple pre-processed groups to form a galaxy cluster could result in
little to no observed mass segregation in the cluster as a whole.  In
other words, if the cluster environment consists of multiple subhaloes
at various cluster-centric radii, while individual subhaloes may show
mass segregation, the total effect of these subhaloes together may
leave the cluster with a relatively uniform radial mass distribution.
\par
\citet{vulcani2014} apply semi-analytic models to the Millenium
Simulation \citep{springel2005} to study galaxy mass functions in
different environments.  Vulcani et al. simulate galaxy mass functions
for three halo masses, $\log(M_\mathrm{halo}/M_\odot) =
\{13.4,\,14.1,\,15.1\}$, as a function of cluster-centric radius.  In
the lowest mass halo they find the mass function depends slightly on
cluster-centric radius, with the innermost regions showing flatter
mass functions at low and intermediate masses.  This trend persists,
but is not as strong at intermediate halo mass.  The highest halo mass
bin shows virtually identical mass function shapes for all
cluster-centric radii.  This result is indicative of measurable mass
segregation for the low- and intermediate-mass haloes, with the
strength of mass segregation decreasing with increasing halo mass.
These simulation trends show excellent agreement with our observed
dependence of mass segregation on halo mass.

\subsection{Reconciling previous results}

In Section~\ref{sec:intro_ms}, we mention previous literature results
which presente vidence both for and against the presence of mass
segregation in groups and clusters.  We argue that the majority of
these results can be reconciled with our two main findings.

\begin{enumerate}[(i)]
  \item Mass segregation is enhanced with the inclusion of low-mass
    galaxies in a sample.

  \item Mass segregation decreases with increasing halo mass, with
    high-mass haloes showing little to no mass segregation.
\end{enumerate}

\noindent
Of the studies mentioned in Section~\ref{sec:intro_ms}, those which
observe no evidence for mass segregation either:  make a mass
completeness cut at intermediate to high stellar mass, or observe this
lack of mass segregation only in high-mass haloes.  Therefore, the
lack of observed mass segregation can potentially be explained through
the lack of low-mass galaxies in the study survey, or the study being
limited to high-halo-mass environments.

\section{Conclusion}
\label{sec:conclusion_ms}

In this Letter, we examine mass segregation trends in the
\citet{yang2007} SDSS DR7 groups for various stellar and halo mass
cuts.  WE show that a small, but significant, amount of mass
segregation is present in these groups.  This mass segregation shows
consistent trends, with lower stellar mass samples showing stronger
mass segregation, and galaxies in large haloes showing little to no
mass segregation.
\par
The magnitude of mass segregation we measure, especially in high-mass
haloes, is potentially indicative of dynamical friction not acting
efficiently.  We discuss previous literature to provide possible
explanations for the observed trends, showing that our observed trends
with halo mass agree with prior results.  Further work with
hydrodynamic simulations would be helpful to further constrain the
important mechanisms responsible for our observed mass trends and the
lack of mass segregation in high-mass haloes.

\section{Acknowledgements}
\label{acknowledgements_ms}

We thank the anonymous referee for their various helpful comments and
suggestions.  IDR and LCP thank the National Science and Engineering
Research Council of Canada for funding.  We thank X. Yang et al. for
making their SDSS DR7 group catalogue public, L. Simard et al. for the
publication of their SDSS DR7 morphology catalogue, and the NYU-VAGC
team for the publication of their SDSS DR7 catalogue.  This research
would not have been possible without these public catalogues.
\par
Funding for the SDSS has been provided by the Alfred P. Sloan
Foundation, the Participating Institutions, the National Science
Foundation, the US Department of Energy, the National Aeronautics and
Space Administration, the Japanese Monbukagakusho, the Max Planck
Society, and the Higher Education Funding Council for England.  The
SDSS website is http://www.sdss.org/.

% Bibliography
%
\bibliographystyle{apj}
\bibliography{masters-thesis}
